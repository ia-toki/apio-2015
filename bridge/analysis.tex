\documentclass[a4paper]{article}

\usepackage{hyperref}
\usepackage{graphicx}
\usepackage{amsmath}
\usepackage{amsthm}
\usepackage{listings}
\usepackage{mathtools}


\newtheorem{algorithm_def}{Algorithm}[section]
\newtheorem{definition}{Definition}[section]
\newtheorem{lemma}{Lemma}[section]
\newtheorem{theorem}{Theorem}[section]
\newtheorem{corollary}{Corollary}[section]

%opening
\title{Ashar Pacific Irvan Olympiad Editorial}

\author{
  Jahja, Irvan\\
  \texttt{irvan@comp.nus.edu.sg}
  \and
  Fufu-chan\\
  \texttt{fufufufufu@ui.edu}
}

\begin{document}

\maketitle

\section{Bridge}

Suppose we build two bridges at position $L$ and $R$ (wlog $L < R$). For a given interval
$(P_s, P_e)$ (again wlog $P_s < P_e$), how do we determine which bridge to use? We can only bridge $L$
iff

\begin{equation}
  \label{dist}
|P_s - L| + |P_e - L| < |P_s - R| + |P_e - R|
\end{equation}

(If they are equal, we can use either bridge).

Using Equation~\ref{dist}, we will now prove the following

\begin{lemma}
\label{lem}
  It is optimal to use the bridge at $L$ if
  \begin{equation}
    \label{eqlemma}
    P_s + P_e < L+R
  \end{equation}
  And it is optimal to use the bridge at $R$ if
  \begin{equation}
    \label{eqlemma}
    P_s + P_e > L+R
  \end{equation}
\end{lemma}

Note that this does not imply the invese. That is, optimality of using bridge
at $L$ or $R$ does not imply the equations.

\begin{proof}

To prove Lemma~\ref{lem}, we need to prove that we can derive
Equation~\ref{dist} from Equation~\ref{eqlemma}. We do so case by case.
Denote the interval $(P_s, P_e)$ with

\begin{verbatim}
|---------|
\end{verbatim}

\subsection{Case 1}

\begin{verbatim}
|---------| L R
\end{verbatim}

That is, we have $P_e \le L$.

In this case, Equation~\ref{dist} expands to

\begin{eqnarray*}
  |P_s - L| + |P_e - L| &<& |P_s - R| + |P_e - R| \\
  2L - (P_s + P_e) &<& 2R - (P_s - P_e) \\
  2L &<& 2R
\end{eqnarray*}

Which is always true. So, we need to prove that in this case,
$P_s + P_e < L+R$ always holds. Since we have $P_e \le L$:

\begin{eqnarray*}
  P_s + P_e &\le P_e + P_e & \text{(Since $P_s \le P_e$)} \\
            &\le 2L & \text{(Since $P_e \le L$)} \\
            &< L + R & \text{(Since $L < R$)}
\end{eqnarray*}

\subsection{Case 2}

\begin{verbatim}
L R |---------|
\end{verbatim}

That is, we have $P_s \ge R$. We prove similar to Case 1.
Equation~\ref{dist} expands to

\begin{eqnarray*}
  |P_s - L| + |P_e - L| &<& |P_s - R| + |P_e - R| \\
  (P_s + P_e) - 2L &<& (P_s - P_e) - 2R\\
  2L &>& 2R
\end{eqnarray*}

Which is always false. So, we need to prove that in this case,
$P_s + P_e < L+R$ never hold. Since we have $P_s \ge R$:

\begin{eqnarray*}
  P_s + P_e &\ge P_s + P_s & \text{(Since $P_s \le P_e$)} \\
            &\ge 2R & \text{(Since $P_s \ge R$)} \\
            &> L + R & \text{(Since $L < R$)}
\end{eqnarray*}

\subsection{Case 3}

\begin{verbatim}
   |---------|
L       R
\end{verbatim}

Thus, we have $L \le P_s \le R \le P_e$.

Equation~\ref{dist} expands to

\begin{eqnarray*}
  |P_s - L| + |P_e - L| &<& |P_s - R| + |P_e - R| \\
  (P_s + P_e) - 2L &<& R - P_s + P_e - R \\
  (P_s + P_e) - 2L &<& P_e - P_s \\
  2P_s &<& 2L
\end{eqnarray*}

Which is always false. So, we need to prove that in this case,
$P_s + P_e < L+R$ never hold:

\begin{eqnarray*}
  P_s + P_e &\ge L + P_e & \text{(Since $L \le P_s$)} \\
            &\ge L + R & \text{(Since $R \le P_e$)}
\end{eqnarray*}

\subsection{Case 4}

\begin{verbatim}
   |---------|
        L         R
\end{verbatim}

Thus, we have $P_s \le L \le P_e \le R$.

Equation~\ref{dist} expands to

\begin{eqnarray*}
  |P_s - L| + |P_e - L| &<& |P_s - R| + |P_e - R| \\
  L - P_s + P_e - L &<& 2R - P_s - P_e \\
  2P_e &<& 2R
\end{eqnarray*}

Which is always true. So, we need to prove that in this case,
$P_s + P_e < L+R$ always hold:

\begin{eqnarray*}
  P_s + P_e &\le L + P_e & \text{(Since $P_s \le L$)} \\
            &\le L + R & \text{(Since $P_e \le R$)}
\end{eqnarray*}

\subsection{Case 5}

\begin{verbatim}
   |---------|
      L   R
\end{verbatim}

Thus, we have $P_s \le L \le R \le P_e$.

Equation~\ref{dist} expands to

\begin{eqnarray*}
  |P_s - L| + |P_e - L| &<& |P_s - R| + |P_e - R| \\
  L - P_s + P_e - L &<& R - P_s + P_e - R \\
  0 &<& 0
\end{eqnarray*}

Which is always false, and in particular, it implies that the distance
of using either of the bridges is the same. Thus, using any of the two
bridges is always optimal.

\subsection{Case 6}

\begin{verbatim}
   |---------|
L               R
\end{verbatim}

Thus, we have $L \le P_s \le P_e \le R$.

From Equation~\ref{eqlemma}:
\begin{eqnarray*}
  P_s + P_e &<& L + R \\
  2(P_s + P_e) &<& 2L + 2R \\
  (P_s + P_e) - 2L &<& 2R - (P_s + P_e) \\
  |P_s - L| + |P_e - L| &<& |P_s - R| + |P_e - R|
\end{eqnarray*}

Which is equation~\ref{dist}.

\end{proof}

\section{Bridge pt.2 }
Thus, there exists a way to separate the intervals into those which should
go to the left bridge and those which should go to the right bridge.
Sort the intervals by $P_s + P_e$, and for every prefix of this sequence
of intervals, try to find the optimal placement of the bridge.

One way to do this in $O(N \log^2 N)$: we keep track of a data structure
to count the cost of placing a bridge in each possible position (initially,
all are zero). We add the bridges one by one. When we add a bridge $P_s, P_e$,
we are actually adding to each position $x$, $|P_s - x|$ and $|P_e - x|$.
$f(x) = |A - x|$, though, is convex, so the sum over all these convex
functions is a convex function. So we can ternary search over the sum of
these convex functions to find the minimum in $\log^2 (n)$ time.



\end{document}


