\documentclass[a4paper,11pt]{article}
\title{Partition}
\begin{document}
\maketitle
\section{Subtask 1}
Constraint : $1\leq N \leq 20; 1 \leq A \leq B \leq N; 0 \leq Y_i \leq 1,000,000,000$\newline
Due to the small N of this subtask, it is possible to do a brutal complete search. We can try whether to insert a "separator" between each consecutive. Among all those possible "separator" positions, we only consider those with more than or equal to $A$ and less than or equal to $B$ groups, and find the minimum OR-SUM. The complexity for this solution is $O(N2^N)$.

\paragraph{The solutions for the next subtasks require an understanding of Dynamic Programming}
\section{Subtask 2}
Constraint : $1\leq N \leq 50; 1 \leq A \leq B \leq \min(20,N); 0 \leq Y_i \leq 10$\newline
We would like to create a dynamic programming table with three parameter, $curpos$, $curOR$, and $curpartition$. \newline
$dp[curpos][curOR][curpartition]$ will be $true$ if and only if there exists a partition of the first $curpos$ elements with $curpartition$ groups and the OR-sum is $curOR$. It can be computed by
\[dp[curpos][curOR][curpartition] =\]
\[\bigvee\limits_{i=curpos}^{N}dp[i+1][curOR | sum[curpos,i]][curpartition+1]\]
where $sum[x,y]$ is the sum of the age of sclupture $x$ to $y$ inclusive. \newline
After computing the dp table, we are finding the minimum number $X$ such that there exists $P$, $A \leq P \leq B$ where $dp[N][X][P]$ is true. Since the OR-SUM of any partition is less than the sum of all elements, we can loop for all possible values for $X$. $curOR$ can be as worse as $O(Y_iN)$. Therefore, the complexity for this solution is $O(NY_iNNN + NY_iN) = O(N^4Y_i)$

\section{Subtask 3}
Constraint : $1\leq N \leq 100; 1 = A \leq B \leq N; 0 \leq Y_i \leq 20$\newline
$A = 1$ means we can try to use as few groups as possible for any fixed $curOR$ and $curpos$. We make a minor modification to our solution from subtask 2. We remove the third parameter, left with only $curpos$ and $curOR$. For this subtask, $dp[curpos][curOR]$ will return the fewest number of groups to make the partition of the first $curpos$ elements with the OR-sum is $curOR$. It can be computed by
\[dp[curpos][curOR] = 1 + \min_{i=curpos}^{N}dp[i+1][curOR | sum[curpos,i]] \]
After computing the dp table, we are finding the minimum number $X$ such that $dp[N][X] \leq B$. The complexity for this solution is $O(NY_iNN + NY_i) = O(N^3Y_i)$

\section{Subtask 4}
Constraint : $1\leq N \leq 100; 1 \leq A \leq B \leq N; 0 \leq Y_i \leq 1,000,000,000$\newline
Since we are looking for minimum OR-SUM, we can greedily choose the most significant bit of the OR-SUM to be zero, if it is possible. We iterate $k$ from $\log(Y_iN)$ to $0$. For any $k$, we try whether it is possible to set the $k$-th bit of the OR-SUM to be zero, regardless of the value of the bit in the less significant bit position. We can use the dynamic programming method explained in subtask 2, but we drop the $curOR$ parameter. The complexity for this solution is $O(\log(Y_iN)NNN) = O(N^3\log(Y_iN))$

\section{Subtask 5}
Constraint : $1\leq N \leq 2,000; 1 = A \leq B \leq N; 0 \leq Y_i \leq 1,000,000,000$\newline
We can solve this subtask by combining the idea from subtask 4 and subtask 3. From subtask 3, we can drop the $curpartition$ parameter by computing the minimum number of partition in the DP table instead. From subtask 4, we can drop the $curOR$ parameter. By combining both ideas, we can have a dp table where the parameter is only $curpos$. The complexity for this solution is $O(\log(Y_iN)NN) = O(N^2\log(Y_iN))$

\end{document}
